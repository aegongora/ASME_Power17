% Dorien O. Villafranco & Aldair E. Gongora
% ASME Power and Energy Journal/Conference Submission
% Abstract 

\documentclass[12pt]{article}
\usepackage{geometry}
\usepackage{amsmath}
\usepackage{amssymb}
\usepackage{enumitem}
\usepackage{fancyhdr}
\usepackage{tikz}
\usetikzlibrary{trees}
\usepackage{amsmath}
\usepackage{listings}
\usepackage{caption}

\begin{document}
\noindent
\textbf{Title:} \\

Energy from sugarcane bagasse in Belize: An economic assessment and overview of potential opportunities. \\

\noindent\textbf{Abstract:}\\
Sugarcane is an important crop for the economy of Belize. [Insert statistic about sugar cane use in economy eg. amount of GDP etc etc.] Since [insert year], sugarcane bagasse has been used for electric energy production. Currently, the single co-generation plant in the country contributes about $15\%$ of electricity to the national grid. With energy needs of the country projected to rise at a rate of $4\%$ per annum, and with the costly import of energy, there exists the need to explore the expansion of the co-generation plant's output to the national grid. The aim of this paper is to demonstrate the positive conditions which support such an expansion, and examine the necessary economic, regulatory and technical pathways toward this increase of output. This is done through a discussion of various economic and technical scenarios with recommendations made for the most feasible option. [Insert statistic about sugar cane production in Belize, increase in investment, regulation, could possibly lead to xx percent being supplied to the national grid; also explore the possibility of diversifying the renewable energy being produce -- ethanol?]

















\end{document}

