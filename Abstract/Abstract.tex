% Dorien O. Villafranco & Aldair E. Gongora
% ASME Power and Energy Journal/Conference Submission
% Abstract 

\documentclass[12pt]{article}
\usepackage{geometry}
\usepackage{amsmath}
\usepackage{amssymb}
\usepackage{enumitem}
\usepackage{fancyhdr}
\usepackage{tikz}
\usetikzlibrary{trees}
\usepackage{amsmath}
\usepackage{listings}
\usepackage{caption}

\begin{document}
\noindent
\textbf{Title:} \\
\\Energy from sugarcane bagasse in Belize: An economic assessment and overview of potential opportunities. \\ \\
\noindent\textbf{Abstract:}\\
\\Sugarcane is the most important crop for the economy of Belize. With sugarcane deliveries exceeding 122 thousand metric tons in the fourth quarter of 2015, sugar is Belize's largest contributor to the agricultural sector with exports approximating BZ\$140,000,000 for the year 2015. With commissioning of the Belize Co-generation Energy (Belcogen) plant completed in December 2009, sugarcane bagasse has been used for electric energy production. Currently, the single co-generation plant in the country contributes about $15\%$ of electricity to the national grid. Belize currently imports about $45\%$ of its electric energy needs from Mexico, and to date, this is the country's most reliable energy source. With energy needs of the country projected to rise at a rate of $4\%$ per annum, and with the costly import of energy, there exists the need to explore the expansion of co-generation energy technologies to increase local energy generation output to the national grid. The aim of this paper is to demonstrate the positive conditions which support such an expansion by analyzing the current co-generation technologies in the context of necessary economic, regulatory and technical pathways toward this increase of output. This is done through a discussion of various economic and technical scenarios with recommendations made for the most feasible option. The introduction of a new sugar plant in Belize with projections of annually producing 100,000 metric tons by 2020 and with intentions of introducing co-generation technologies could lead to $20\%$ of local electrical energy being supplied to the national grid. It is projected that a new sugar plant of this size would grow the agricultural gross domestic product of the country by approximately $4\%$. These benefits further support the need for a discussion of co-generation technologies and the associated positive conditions supporting the expansion of the aforementioned technologies. Additionally, it is also relevant to mention the feasibility of alternative renewable energy technologies that could contribute to Belize's national goal of increasing energy efficiency, sustainability, and resilience over the next 30 years. 
\end{document}

